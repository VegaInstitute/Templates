\documentclass[]{vegapresentation}

\title{Sample Presentation}
\subtitle{III International School in Mathematical Finance}
\author{John Smith}
\institute{Vega Institute Foundation}
\date{August 21 -- 28, 2022}

\usepackage[]{lipsum}
\begin{document}
    \maketitle

    \begin{frame}{Text example}
        \lipsum[1]
    \end{frame}

    \begin{frame}{List example}{Bullets}
        Popular models:
        \begin{itemize}
            \item Cox-Ross-Rubinstein;
            \item Bachilier;
            \item Black-Sholes;
            \item Black;
            \item CEV.\footnote{Local volatility model, see Dupire.}
        \end{itemize}
    \end{frame}

    \begin{frame}{List example}{Enum}
        Popular models:
        \begin{enumerate}
            \item Cox-Ross-Rubinstein;
            \item Bachilier;
            \item Black-Sholes;
            \item Black;
            \item CEV.\footnote{Local volatility model, see Dupire.}
        \end{enumerate}
    \end{frame}

    \begin{frame}{Block  and equation example}
        \begin{theorem}[И. Гирсанов]
			Если $\lambda_T = (\lambda_t (\omega))_{t \leq T)}$ таков, что 
			\begin{equation*}
				\E e^{\int_0^T \lambda_t dB_t - \frac{1}{2}\int_0^T \lambda_t^2 dt} = 1
			\end{equation*}
			и
			\begin{equation*}
			 dP_T^\lambda = e^{\int_0^T \lambda_t dB_t - \frac{1}{2}\int_0^T \lambda_t^2 dt} dP_T,
			\end{equation*}
			то процесс 
			\begin{equation*}
				B^\lambda_t = B_t - \int_0^t \lambda_s(\omega) ds, t \leq T
			\end{equation*}
			 является $P_T^\lambda$-броуновским движением.
		\end{theorem}
    \end{frame}
\end{document}